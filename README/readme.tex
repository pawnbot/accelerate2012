\documentclass[12pt,a4paper]{article} 
\usepackage[utf8]{inputenc}
\usepackage[russian]{babel}
\usepackage{amsmath}
\usepackage{amsfonts}
\usepackage{amssymb}
\usepackage{makeidx}
\usepackage[left=3cm, right=3cm, top=2cm, bottom=2cm]{geometry}
\usepackage{graphicx}


\begin{document}
\newcommand{\parw}[2]{\dfrac{\partial #1}{\partial #2}}
\newcommand{\pl}{p_{\lambda}}
\newcommand{\pa}{p_{a}}
\newcommand{\mn}[1]{ \{ {#1} \} }
\newcommand{\pgm}{p_{m, \sigma^2}}
\newcommand{\pma}{p_{M, a}}
\newcommand{\xseqn}{x_1, \ldots, x_n}
\newcommand{\nd}[1]{\frac{1}{\sqrt{2 \pi #1^2}} \exp ( -{\frac{x^2}{2 #1^2}} ) }
\newcommand{\ndm}[2]{\frac{1}{\sqrt{2 \pi #2^2}} \exp ( - {\frac{(x - #1)^2}{2 #2^2}})}
\newcommand{\ndms}[3]{\frac{1}{(\sqrt{2 \pi #2^2})^n} \exp ( - \sum\limits_{i = 1}^{#3}{\frac{(x_i - #1)^2}{2 #2^2}})}
\renewcommand{\le}{\leqslant}
\renewcommand{\ge}{\geqslant}

\binoppenalty=10000
\relpenalty=10000
\newcommand*{\hm}[1]{#1\nobreak\discretionary{}%
{\hbox{$\mathsurround=0pt #1$}}{}}

\begin{titlepage}
\begin{center}
\huge{\textbf{BrozTeam}

\bigskip
\Large{Параллельный алгоритм поиска\\общих подстрок, больших заданной длины}}
\end{center}

\begin{flushright}
\textbf{А.\,С. БОГАТЫЙ}

\textit{Московский государственный университет\\имени М. В. Ломоносова\\механико-математический факультет, 3 курс\\}
e-mail: bogatyia@gmail.com\\
isn: pawnbot

\bigskip
\textbf{И.\,С. БОГАТЫЙ}

\textit{Московский государственный университет\\имени М. В. Ломоносова\\механико-математический факультет, 5 курс\\}
e-mail: bogatyi@gmail.com\\
isn: loken17
\\[70pt]
\end{flushright}

\end{titlepage}


\section{Введение}
В конкурсе \textbf{Accelerate 2012} компании \textbf{Intel} была предложена задача поиска всех общих подстрок у двух строк, эти подстроки должны удовлетворять некоторым критериям: подстроку нельзя расширить и сдвинуть вправо.
\section{Описание решения поставленной задачи}
\subsection{Наивный алгоритм}
Самым простым алгоритмом решения поставленной задачи является предложенный организаторами квадратный алгоритм, он же является оптимальным, если матрица $L[i][j]$ помещается в кэш процессора. Так как при вычислении $i$-ой строки мы используем только $i - 1$ строку, то достаточно хранить только 2 последние строки матрицы, уменьшив тем самым расход памяти. Распараллеливание здесь простое - каждому потоку своя пара строк для сравнения.

\subsection{Асимптотически оптимальный алгоритм}
Асимптотически оптимальный алгоритм в данной задаче - это построить суффиксное дерево транспонированной первой строки и суффиксное дерево транспонированной второй строки, потом объединить эти два дерева (тяжело реализуемая операция). Тогда $L[i][j]$ - это наименьший общий предок $i$-го и $j$-го суффикса. Причём для каждого суффикса $i$ нужно перебирать только те суффиксы $j$, у которых предок на глубине $m$(минимальная длина совпадения) совпадает с предком $i$ на глубине $m$. Построить суффиксное дерево можно за $O(N)$ с помощью алгоритма Укконена, где $N$ - длина строки, слить два дерева можно за $O(N+M)$. Дальше мы тратим $O(K)$ операций, где $K$ - число последовательностей в ответе (без учета того, что последовательность нельзя сдвинуть вправо).
Получаем линейный алгоритм за $O(N + M + K)$. Увы, несмотря на маленький размер алфавита, у алгоритма Укконена неприлично большая константа и этот алгоритм нельзя распараллелить. Поэтому для достижения нормального времени работы мы использовали следующий эвристический алгоритм:

\subsection{Параллельный алгоритм решения}
Строим суффиксное дерево меньшей из двух строк. Перебираем все индексы большей строки и спускаемся вниз по дереву на 
$m$. Дальше выписываем все суффиксы, которые ниже вершины, до которой мы дошли. В итоге мы для индекса $i$ получили все индексы $j$, такие что $L[i][j] \geq m$. Причём если такая вершина на глубине $m$ существует, то пройдя по суффиксной ссылке из этой вершины мы попадём в вершину на глубине $m - 1$ для индекса $i - 1$. Если перебирать индексы $i$ последовательно, начиная с конца, а все такие индексы $j$ хранить в $map$, то переход от $i + 1$ к $i$ примет вид:
\\
1) пройти по суффиксной ссылке из вершины для индекса $i + 1$.
\\
2) получить все индексы $j$, что $L[i][j] \geq m$
\\
3) для всех таких $j$ посмотреть если $L[i+1][j+1] \neq 0$, то $L[i][j] = L[i+1][j+1]-1$, иначе вычислить $L[i][j]$. 
\\
4) Добавить в ответ те индексы $j$, которые удовлетворяют фильтрам задачи (нельзя сдвинуть вправо и расширить).
\\
Этот алгоритм легко можно распараллелить, например, каждому потоку можно дать свой кусок индексов $first...last$ для проверки. Увы, алгоритм всё равно остаётся весьма медленным. Поэтому применяется следующая оптимизация - для всех возможных подстрок длины $k$ вычисляется их положение при спуске с дерева. Тогда первый прыжок делается не на $1$, а сразу на $k$. Естественно, возникает желание взять $k$ побольше, но одновременно не слишком большое, поскольку на это используется $4^k$ памяти(map использовать нельзя, поскольку он погубит производительность).
\\
Изначально было взято $k=8$, программа значительно ускорилась. Потом $k=12$ и несмотря на то, что индексы $4^{12}$ строк вычислялись моментально, сама программа замедлилась. В дальнейшем стало понятно, что при $k=8$ потоки постоянно лезут за индексом в кэш и всё работает моментально, а при $k=12$ они постоянно промахиваются мимо кэша и лезут в оперативную память. Оптимальным значением для $Xeon$-ов оказалось $k=10$.

\subsection{Особенности практической реализации}
Решение было написано на языке $C++$. В качестве библиотеки для распараллеливания использовались \texttt{pthreads}, поскольку несмотря на свою громоздкость обладают максимальной производительностью.
Важными моментами реализации мы считаем:
\\
1) Отказ от векторов и динамических выделений памяти в реализации суффиксного дерева.
\\
2) Не рекурсивная версия алгоритма Укконена.
\\
3) Использование \texttt{bfs}, вместо \texttt{dfs} для обхода дерева. Рекурсивный \texttt{dfs} может упасть из-за нехватки стэка.

\subsection{Сравнение времени работы программы при разном количестве используемых процессоров}
Для тестирования ускорения использовали команду:
\\
\begin{verbatim}
time ./run X 16 test_input_3.fa (дальше идут все хромосомы, которые были в архиве)
\end{verbatim}
Тесты проводились на системе с 2 x \texttt{Xeon E5645}.
\\
\begin{tabular}{|c|c|c|}
\hline 
$\text{Количество потоков}$ & $\text{Время (sec)}$ & $\text{Ускорение} $ \\ 
\hline 
$1$ & $58.19$ &  $1.00$ \\ 
\hline 
$2$ & $31.12$  &  $1.86$ \\ 
\hline 
$4$ & $20.06$  &  $2.90$ \\ 
\hline 
$12$ & $13.19$ & $4.41$ \\ 
\hline
\end{tabular}

\begin{thebibliography}{99}
\bibitem{gusfield}
\textit{Dan Gusfield} Algorithms on Strings, Trees, and Sequences.
\bibitem{bogachev_book}
\textit{Богачев К.\,Ю. } Основы параллельного программирования. М.: Бином, 2003.
\end{thebibliography}

\end{document}


